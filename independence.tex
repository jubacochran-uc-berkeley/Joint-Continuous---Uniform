\documentclass[12pt]{amsart}

\pagestyle{empty}

\usepackage{amsmath,amssymb,amsfonts,amsthm}
\usepackage{enumerate}% http://ctan.org/pkg/enumerate
\usepackage{eucal}

\usepackage{graphicx,psfrag} %only include if using pictures

\usepackage{ifthen} %only include if using conditional package

%\usepackage{xymatrix}


%keep this...%%%%%%%%%%%%%%%%%%%%%%%%%%%%%%%%%%%%%%%%%%%%%%%%%%%%
\vfuzz2pt % Don't report over-full v-boxes if over-edge is small
\hfuzz2pt % Don't report over-full h-boxes if over-edge is small
%%%%%%%%%%%%%%%%%%%%%%%%%%%%%%%%%%%%%%%%%%%%%%%%%%%%%%%%%%%%%%%%%

%Side Margins
\evensidemargin 0.1 in \oddsidemargin 0.1 in

%Paragraph Size
\parindent 24pt


%size of page
\textheight 9.6 in \textwidth 6.2 in
\baselineskip 9.6 in \topmargin 0.005 in


% ***********************************************************************
\newcommand{\Normalstretch}{1.0} %change to define space between lines
\renewcommand{\baselinestretch}{\Normalstretch}
\newenvironment{singlespace}
{\renewcommand{\baselinestretch}{1} \small \normalsize}
{\renewcommand{\baselinestretch}{\Normalstretch} \small \normalsize}
% ***********************************************************************
\newcommand{\baseenvskip}{\baselineskip 2mm}

% Standard Notation for Theorems and Lemmas
\newtheorem{mydef}{Definition}

\newtheorem{thm}{Theorem}

\newtheorem{lemma}[thm]{Lemma}

\newtheorem{claim}[thm]{Claim}

\newtheorem{proposition}[thm]{Proposition}

\newtheorem{conjecture}[thm]{Conjecture}

\newtheorem{corollary}[thm]{Corollary}

\theoremstyle{remark}
\newtheorem{rmk}{Remark}[section]
\newenvironment{remark}{\begin{rmk}\rm\baseenvskip}{\end{rmk}}
\newtheorem{definition}[rmk]{Definition}

%How numbering is defined in articles (standard)
\numberwithin{equation}{section} \numberwithin{thm}{section}
\numberwithin{rmk}{section} \numberwithin{figure}{section}

% ************************ space ************************************
\newcommand{\hhs}[1]{\hspace{#1mm}}
\newcommand{\hs}{\hspace{5mm}}
\newcommand{\vp}{\vspace{1mm}}
\newcommand{\vs}{\vspace{5mm}}
\newcommand{\jl}{$\frac{}{}$} %User defined for empty symbol to jump line
% ********************** newcommand *********************************
\newcommand{\mbf}[1]{\mbox{\boldmath $#1$}}
\newcommand{\hb}[1]{\hspace{-#1 mm}}
\newcommand{\ds}{\displaystyle}
\newcommand{\QED}{\hfill $\Box$}
% *********************** frequently used math symbols from AMS *******
\newcommand{\norm}[1]{\left\Vert#1\right\Vert}
\newcommand{\abs}[1]{\left\vert#1\right\vert}
\newcommand{\set}[1]{\left\{#1\right\}}

% ************** Some frequently used symbols - User defined ***********
% ************** Also Called MACROS ************************************

\newcommand{\N}{\mathbb{N}}
\newcommand{\Q}{\mathbb{Q}}
\newcommand{\Hv}{\mathcal{H}_v}
\newcommand{\h}{\mathcal{H}}
\newcommand{\Ov}{\mathcal{O}_v}
\newcommand{\F}{\mathcal{F}}
\newcommand{\Rn}{R^n}
\newcommand{\C}{\mathbb{C}}
\newcommand{\Ce}{\widehat{\mathbb{C}}}
\newcommand{\si}{\sigma}
\newcommand{\Cn}{\mathbb{C}^n}
\newcommand{\Z}{\mathbb{Z}}
\newcommand{\p}{\partial}
\newcommand{\ep}{\varepsilon}
\newcommand{\D}{\delta}
\newcommand{\eps}{\varepsilon}
\newcommand{\finv}{f^{-1}}
\newcommand{\im}{\imath}
\newcommand{\ga}{\gamma}
\newcommand{\ze}{\zeta}
\newcommand{\fee}{\varphi}
\newcommand{\noi}{\noindent}
%
% *******************************************************************
% At First run of Template, only modify from here below........ *****
% *******************************************************************


\begin{document}

\title{}

% ******************** ABSTRACT *************************************
%\begin{abstract}
%Insert abstract
%\end{abstract}
% Must be present so the above information is displayed.
% *******************************************************************
% **** Begin Typing your work from here below ***********************
% *******************************************************************


\textbf{
\section{{Q2 Functions and Independence}}
}

\vspace{2em}
Suppose $\text{X}$ is a discrete random variable and $\text{g:} \mathbb{R}$→$\mathbb{R}$ is a function from the real numbers to the real numbers.
\vspace{2em}

Q2.1 
\subsubsection{Functions and Independence}
Prove that if $\text{X}$ and $\text{g(X)}$ are independent, then $\text{g(X)}$must be a constant (a random variable with only a single point in its support).

\vspace{2em}


\begin{proof}

\begin{thm}[Theorem 1.1.15: Independence of Events]
Events $A,B, \in S$ are independent if $P(A\cap B)$ = $P(A)P(B)$
\end{thm}

\begin{thm}[Theorem 1.1.9: Multiplicative Law of Probability]
A and B are independent $A \Leftrightarrow B $ P(A$\mid$B)P(B) = P(A)P(B)
                                                           $ \Leftrightarrow  $ P(A$\mid$B) = P(A). 
 \end{thm}
\begin{thm}[Theorem 1.3.17: Implications of Independence]
Let X and Y be either two discrete random variables with joint PMF f or two jointly continuous random variables with joint PDF f. Then the following statements are equivalent. (Foundations of Agnostic Statistics pg.39)

\begin{itemize}
    \item \( X \perp\!\!\!\perp Y \)
    \item \( \forall x, y \in \mathbb{R},\ f(x, y) = f_X(x) f_Y(y) \)
    \item \( \forall x \in \mathbb{R} \) and \( \forall y \in \text{Supp}[Y],\ f_{X \mid Y}(x \mid y) = f_X(x) \)
    \item \( \forall D, E \subseteq \mathbb{R},\ \text{the events } \{ X \in D \} \text{ and } \{ Y \in E \} \text{ are independent} \)
    \item For all functions \( g \) of \( X \) and \( h \) of \( Y \), \( g(X) \perp\!\!\!\perp h(Y) \)
\end{itemize}
\end{thm}
\vspace{2em}

Now suppose \( X \) and \( g(X) \) are independent.

These three theorems can map our understanding to g(X) being a constant with whose support consists of a single value. If there is not information shared between these variables then they must be independent and g(X) must be a constant. 




\end{thm}


\vspace{2em}



\end{proof}

\end{document}